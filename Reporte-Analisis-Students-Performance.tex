% Options for packages loaded elsewhere
\PassOptionsToPackage{unicode}{hyperref}
\PassOptionsToPackage{hyphens}{url}
%
\documentclass[
]{article}
\usepackage{amsmath,amssymb}
\usepackage{iftex}
\ifPDFTeX
  \usepackage[T1]{fontenc}
  \usepackage[utf8]{inputenc}
  \usepackage{textcomp} % provide euro and other symbols
\else % if luatex or xetex
  \usepackage{unicode-math} % this also loads fontspec
  \defaultfontfeatures{Scale=MatchLowercase}
  \defaultfontfeatures[\rmfamily]{Ligatures=TeX,Scale=1}
\fi
\usepackage{lmodern}
\ifPDFTeX\else
  % xetex/luatex font selection
\fi
% Use upquote if available, for straight quotes in verbatim environments
\IfFileExists{upquote.sty}{\usepackage{upquote}}{}
\IfFileExists{microtype.sty}{% use microtype if available
  \usepackage[]{microtype}
  \UseMicrotypeSet[protrusion]{basicmath} % disable protrusion for tt fonts
}{}
\makeatletter
\@ifundefined{KOMAClassName}{% if non-KOMA class
  \IfFileExists{parskip.sty}{%
    \usepackage{parskip}
  }{% else
    \setlength{\parindent}{0pt}
    \setlength{\parskip}{6pt plus 2pt minus 1pt}}
}{% if KOMA class
  \KOMAoptions{parskip=half}}
\makeatother
\usepackage{xcolor}
\usepackage[margin=1in]{geometry}
\usepackage{color}
\usepackage{fancyvrb}
\newcommand{\VerbBar}{|}
\newcommand{\VERB}{\Verb[commandchars=\\\{\}]}
\DefineVerbatimEnvironment{Highlighting}{Verbatim}{commandchars=\\\{\}}
% Add ',fontsize=\small' for more characters per line
\usepackage{framed}
\definecolor{shadecolor}{RGB}{248,248,248}
\newenvironment{Shaded}{\begin{snugshade}}{\end{snugshade}}
\newcommand{\AlertTok}[1]{\textcolor[rgb]{0.94,0.16,0.16}{#1}}
\newcommand{\AnnotationTok}[1]{\textcolor[rgb]{0.56,0.35,0.01}{\textbf{\textit{#1}}}}
\newcommand{\AttributeTok}[1]{\textcolor[rgb]{0.13,0.29,0.53}{#1}}
\newcommand{\BaseNTok}[1]{\textcolor[rgb]{0.00,0.00,0.81}{#1}}
\newcommand{\BuiltInTok}[1]{#1}
\newcommand{\CharTok}[1]{\textcolor[rgb]{0.31,0.60,0.02}{#1}}
\newcommand{\CommentTok}[1]{\textcolor[rgb]{0.56,0.35,0.01}{\textit{#1}}}
\newcommand{\CommentVarTok}[1]{\textcolor[rgb]{0.56,0.35,0.01}{\textbf{\textit{#1}}}}
\newcommand{\ConstantTok}[1]{\textcolor[rgb]{0.56,0.35,0.01}{#1}}
\newcommand{\ControlFlowTok}[1]{\textcolor[rgb]{0.13,0.29,0.53}{\textbf{#1}}}
\newcommand{\DataTypeTok}[1]{\textcolor[rgb]{0.13,0.29,0.53}{#1}}
\newcommand{\DecValTok}[1]{\textcolor[rgb]{0.00,0.00,0.81}{#1}}
\newcommand{\DocumentationTok}[1]{\textcolor[rgb]{0.56,0.35,0.01}{\textbf{\textit{#1}}}}
\newcommand{\ErrorTok}[1]{\textcolor[rgb]{0.64,0.00,0.00}{\textbf{#1}}}
\newcommand{\ExtensionTok}[1]{#1}
\newcommand{\FloatTok}[1]{\textcolor[rgb]{0.00,0.00,0.81}{#1}}
\newcommand{\FunctionTok}[1]{\textcolor[rgb]{0.13,0.29,0.53}{\textbf{#1}}}
\newcommand{\ImportTok}[1]{#1}
\newcommand{\InformationTok}[1]{\textcolor[rgb]{0.56,0.35,0.01}{\textbf{\textit{#1}}}}
\newcommand{\KeywordTok}[1]{\textcolor[rgb]{0.13,0.29,0.53}{\textbf{#1}}}
\newcommand{\NormalTok}[1]{#1}
\newcommand{\OperatorTok}[1]{\textcolor[rgb]{0.81,0.36,0.00}{\textbf{#1}}}
\newcommand{\OtherTok}[1]{\textcolor[rgb]{0.56,0.35,0.01}{#1}}
\newcommand{\PreprocessorTok}[1]{\textcolor[rgb]{0.56,0.35,0.01}{\textit{#1}}}
\newcommand{\RegionMarkerTok}[1]{#1}
\newcommand{\SpecialCharTok}[1]{\textcolor[rgb]{0.81,0.36,0.00}{\textbf{#1}}}
\newcommand{\SpecialStringTok}[1]{\textcolor[rgb]{0.31,0.60,0.02}{#1}}
\newcommand{\StringTok}[1]{\textcolor[rgb]{0.31,0.60,0.02}{#1}}
\newcommand{\VariableTok}[1]{\textcolor[rgb]{0.00,0.00,0.00}{#1}}
\newcommand{\VerbatimStringTok}[1]{\textcolor[rgb]{0.31,0.60,0.02}{#1}}
\newcommand{\WarningTok}[1]{\textcolor[rgb]{0.56,0.35,0.01}{\textbf{\textit{#1}}}}
\usepackage{graphicx}
\makeatletter
\newsavebox\pandoc@box
\newcommand*\pandocbounded[1]{% scales image to fit in text height/width
  \sbox\pandoc@box{#1}%
  \Gscale@div\@tempa{\textheight}{\dimexpr\ht\pandoc@box+\dp\pandoc@box\relax}%
  \Gscale@div\@tempb{\linewidth}{\wd\pandoc@box}%
  \ifdim\@tempb\p@<\@tempa\p@\let\@tempa\@tempb\fi% select the smaller of both
  \ifdim\@tempa\p@<\p@\scalebox{\@tempa}{\usebox\pandoc@box}%
  \else\usebox{\pandoc@box}%
  \fi%
}
% Set default figure placement to htbp
\def\fps@figure{htbp}
\makeatother
\setlength{\emergencystretch}{3em} % prevent overfull lines
\providecommand{\tightlist}{%
  \setlength{\itemsep}{0pt}\setlength{\parskip}{0pt}}
\setcounter{secnumdepth}{-\maxdimen} % remove section numbering
\usepackage{bookmark}
\IfFileExists{xurl.sty}{\usepackage{xurl}}{} % add URL line breaks if available
\urlstyle{same}
\hypersetup{
  pdftitle={Análisis del Rendimiento Escolar},
  pdfauthor={América Itzel Reyes Alatorre},
  hidelinks,
  pdfcreator={LaTeX via pandoc}}

\title{Análisis del Rendimiento Escolar}
\author{América Itzel Reyes Alatorre}
\date{}

\begin{document}
\maketitle

Introducción: Este análisis explora cómo diversos factores
socioeducativos influyen en el rendimiento académico de estudiantes en
las áreas de matemáticas, lectura y escritura. El estudio se basa en el
dataset ``Students Performance in Exams'' disponible en Kaggle, que
contiene información de 1,000 estudiantes.

Objetivos principales: -Analizar el impacto del género, nivel educativo
de los padres y tipo de almuerzo en los puntajes académicos

-Identificar diferencias significativas entre grupos

-Visualizar patrones y relaciones clave en los datos

\begin{Shaded}
\begin{Highlighting}[]
\CommentTok{\# Cargar librerías}
\FunctionTok{library}\NormalTok{(dplyr)}
\end{Highlighting}
\end{Shaded}

\begin{verbatim}
## 
## Adjuntando el paquete: 'dplyr'
\end{verbatim}

\begin{verbatim}
## The following objects are masked from 'package:stats':
## 
##     filter, lag
\end{verbatim}

\begin{verbatim}
## The following objects are masked from 'package:base':
## 
##     intersect, setdiff, setequal, union
\end{verbatim}

\begin{Shaded}
\begin{Highlighting}[]
\FunctionTok{library}\NormalTok{(ggplot2)}

\CommentTok{\# 1. Cargar datos}
\NormalTok{data }\OtherTok{\textless{}{-}} \FunctionTok{read.csv}\NormalTok{(}\StringTok{"StudentsPerformance.csv"}\NormalTok{)}

\CommentTok{\# 2. Exploración inicial}
\FunctionTok{summary}\NormalTok{(data) }
\end{Highlighting}
\end{Shaded}

\begin{verbatim}
##     gender          race.ethnicity     parental.level.of.education
##  Length:1000        Length:1000        Length:1000                
##  Class :character   Class :character   Class :character           
##  Mode  :character   Mode  :character   Mode  :character           
##                                                                   
##                                                                   
##                                                                   
##     lunch           test.preparation.course   math.score     reading.score   
##  Length:1000        Length:1000             Min.   :  0.00   Min.   : 17.00  
##  Class :character   Class :character        1st Qu.: 57.00   1st Qu.: 59.00  
##  Mode  :character   Mode  :character        Median : 66.00   Median : 70.00  
##                                             Mean   : 66.09   Mean   : 69.17  
##                                             3rd Qu.: 77.00   3rd Qu.: 79.00  
##                                             Max.   :100.00   Max.   :100.00  
##  writing.score   
##  Min.   : 10.00  
##  1st Qu.: 57.75  
##  Median : 69.00  
##  Mean   : 68.05  
##  3rd Qu.: 79.00  
##  Max.   :100.00
\end{verbatim}

\begin{Shaded}
\begin{Highlighting}[]
\FunctionTok{str}\NormalTok{(data)      }
\end{Highlighting}
\end{Shaded}

\begin{verbatim}
## 'data.frame':    1000 obs. of  8 variables:
##  $ gender                     : chr  "female" "female" "female" "male" ...
##  $ race.ethnicity             : chr  "group B" "group C" "group B" "group A" ...
##  $ parental.level.of.education: chr  "bachelor's degree" "some college" "master's degree" "associate's degree" ...
##  $ lunch                      : chr  "standard" "standard" "standard" "free/reduced" ...
##  $ test.preparation.course    : chr  "none" "completed" "none" "none" ...
##  $ math.score                 : int  72 69 90 47 76 71 88 40 64 38 ...
##  $ reading.score              : int  72 90 95 57 78 83 95 43 64 60 ...
##  $ writing.score              : int  74 88 93 44 75 78 92 39 67 50 ...
\end{verbatim}

\begin{Shaded}
\begin{Highlighting}[]
\CommentTok{\# 3. Manejo de valores faltantes }
\FunctionTok{sum}\NormalTok{(}\FunctionTok{is.na}\NormalTok{(data))  }
\end{Highlighting}
\end{Shaded}

\begin{verbatim}
## [1] 0
\end{verbatim}

\begin{Shaded}
\begin{Highlighting}[]
\CommentTok{\# Opcional: Eliminar filas con NAs }
\NormalTok{data\_clean }\OtherTok{\textless{}{-}} \FunctionTok{na.omit}\NormalTok{(data)}
\end{Highlighting}
\end{Shaded}

A continuación se realizan las transformaciones con dplyr, convirtiendo
variables categóricas a factores, creando igual nuevas variables para su
posterior manipulación.

\begin{Shaded}
\begin{Highlighting}[]
\CommentTok{\# 4. Transformaciones con dplyr}
\NormalTok{data\_clean }\OtherTok{\textless{}{-}}\NormalTok{ data }\SpecialCharTok{\%\textgreater{}\%}
  \FunctionTok{rename}\NormalTok{(}
    \AttributeTok{parental\_education =} \StringTok{\textasciigrave{}}\AttributeTok{parental.level.of.education}\StringTok{\textasciigrave{}}\NormalTok{, }
    \AttributeTok{lunch\_type =}\NormalTok{ lunch                                  }
\NormalTok{  ) }\SpecialCharTok{\%\textgreater{}\%}
  \FunctionTok{mutate}\NormalTok{(}
    \AttributeTok{gender =} \FunctionTok{as.factor}\NormalTok{(gender),}
    \AttributeTok{parental\_education =} \FunctionTok{factor}\NormalTok{(parental\_education,}
                                \AttributeTok{levels =} \FunctionTok{c}\NormalTok{(}\StringTok{"some high school"}\NormalTok{, }\StringTok{"high school"}\NormalTok{, }\StringTok{"some college"}\NormalTok{,}
                                           \StringTok{"associate\textquotesingle{}s degree"}\NormalTok{, }\StringTok{"bachelor\textquotesingle{}s degree"}\NormalTok{, }\StringTok{"master\textquotesingle{}s degree"}\NormalTok{)),}
    \AttributeTok{lunch\_type =} \FunctionTok{as.factor}\NormalTok{(lunch\_type), }
    \AttributeTok{avg\_score =}\NormalTok{ (math.score }\SpecialCharTok{+}\NormalTok{ reading.score }\SpecialCharTok{+}\NormalTok{ writing.score) }\SpecialCharTok{/} \DecValTok{3}
\NormalTok{  )}

\CommentTok{\# 5. Estadísticos por grupo }
\NormalTok{data\_clean }\SpecialCharTok{\%\textgreater{}\%}
  \FunctionTok{group\_by}\NormalTok{(gender) }\SpecialCharTok{\%\textgreater{}\%}
  \FunctionTok{summarise}\NormalTok{(}
    \AttributeTok{mean\_math =} \FunctionTok{mean}\NormalTok{(math.score),}
    \AttributeTok{median\_math =} \FunctionTok{median}\NormalTok{(math.score)}
\NormalTok{  )}
\end{Highlighting}
\end{Shaded}

\begin{verbatim}
## # A tibble: 2 x 3
##   gender mean_math median_math
##   <fct>      <dbl>       <dbl>
## 1 female      63.6          65
## 2 male        68.7          69
\end{verbatim}

\begin{Shaded}
\begin{Highlighting}[]
\NormalTok{data\_clean }\SpecialCharTok{\%\textgreater{}\%}
  \FunctionTok{group\_by}\NormalTok{(lunch\_type) }\SpecialCharTok{\%\textgreater{}\%}
  \FunctionTok{summarise}\NormalTok{(}
    \AttributeTok{mean\_math =} \FunctionTok{mean}\NormalTok{(math.score),}
    \AttributeTok{mean\_reading =} \FunctionTok{mean}\NormalTok{(reading.score),}
    \AttributeTok{mean\_writing =} \FunctionTok{mean}\NormalTok{(writing.score),}
    \AttributeTok{count =} \FunctionTok{n}\NormalTok{()}
\NormalTok{  )}
\end{Highlighting}
\end{Shaded}

\begin{verbatim}
## # A tibble: 2 x 5
##   lunch_type   mean_math mean_reading mean_writing count
##   <fct>            <dbl>        <dbl>        <dbl> <int>
## 1 free/reduced      58.9         64.7         63.0   355
## 2 standard          70.0         71.7         70.8   645
\end{verbatim}

Ahora creamos visualizaciones que exploren relaciones entre variables
clave tomando como referencia el cuestionamiento inicial

\begin{Shaded}
\begin{Highlighting}[]
\CommentTok{\# 6. Visualizaciones}

\FunctionTok{ggplot}\NormalTok{(data\_clean, }\FunctionTok{aes}\NormalTok{(}\AttributeTok{x =}\NormalTok{ parental\_education, }\AttributeTok{y =}\NormalTok{ math.score, }\AttributeTok{fill =}\NormalTok{ gender)) }\SpecialCharTok{+}
  \FunctionTok{geom\_boxplot}\NormalTok{() }\SpecialCharTok{+}
  \FunctionTok{labs}\NormalTok{(}
    \AttributeTok{title =} \StringTok{"Rendimiento en matemáticas por género y educación de los padres"}\NormalTok{,}
    \AttributeTok{x =} \StringTok{"Nivel educativo de los padres"}\NormalTok{,}
    \AttributeTok{y =} \StringTok{"Puntaje de matemáticas"}\NormalTok{,}
    \AttributeTok{fill =} \StringTok{"Género"}
\NormalTok{  ) }\SpecialCharTok{+}
  \FunctionTok{theme\_minimal}\NormalTok{() }\SpecialCharTok{+}
  \FunctionTok{theme}\NormalTok{(}\AttributeTok{axis.text.x =} \FunctionTok{element\_text}\NormalTok{(}\AttributeTok{angle =} \DecValTok{45}\NormalTok{, }\AttributeTok{hjust =} \DecValTok{1}\NormalTok{))}
\end{Highlighting}
\end{Shaded}

\pandocbounded{\includegraphics[keepaspectratio]{Reporte-Analisis-Students-Performance_files/figure-latex/unnamed-chunk-3-1.pdf}}

\begin{Shaded}
\begin{Highlighting}[]
\FunctionTok{ggplot}\NormalTok{(data\_clean, }\FunctionTok{aes}\NormalTok{(}\AttributeTok{x =}\NormalTok{ math.score, }\AttributeTok{y =}\NormalTok{ reading.score, }\AttributeTok{color =}\NormalTok{ lunch\_type)) }\SpecialCharTok{+}
  \FunctionTok{geom\_point}\NormalTok{(}\AttributeTok{alpha =} \FloatTok{0.6}\NormalTok{) }\SpecialCharTok{+}
  \FunctionTok{labs}\NormalTok{(}
    \AttributeTok{title =} \StringTok{"Relación entre puntajes de matemáticas y lectura"}\NormalTok{,}
    \AttributeTok{x =} \StringTok{"Matemáticas"}\NormalTok{,}
    \AttributeTok{y =} \StringTok{"Lectura"}\NormalTok{,}
    \AttributeTok{color =} \StringTok{"Tipo de almuerzo"}
\NormalTok{  ) }\SpecialCharTok{+}
  \FunctionTok{scale\_color\_manual}\NormalTok{(}\AttributeTok{values =} \FunctionTok{c}\NormalTok{(}\StringTok{"red"}\NormalTok{, }\StringTok{"blue"}\NormalTok{))  }
\end{Highlighting}
\end{Shaded}

\pandocbounded{\includegraphics[keepaspectratio]{Reporte-Analisis-Students-Performance_files/figure-latex/unnamed-chunk-3-2.pdf}}

\begin{Shaded}
\begin{Highlighting}[]
\FunctionTok{ggplot}\NormalTok{(data\_clean, }\FunctionTok{aes}\NormalTok{(}\AttributeTok{x =}\NormalTok{ lunch\_type, }\AttributeTok{y =}\NormalTok{ math.score, }\AttributeTok{fill =}\NormalTok{ gender)) }\SpecialCharTok{+}
  \FunctionTok{geom\_boxplot}\NormalTok{() }\SpecialCharTok{+}
  \FunctionTok{labs}\NormalTok{(}
    \AttributeTok{title =} \StringTok{"Rendimiento en matemáticas por tipo de almuerzo y género"}\NormalTok{,}
    \AttributeTok{x =} \StringTok{"Tipo de almuerzo"}\NormalTok{,}
    \AttributeTok{y =} \StringTok{"Puntaje de matemáticas"}\NormalTok{,}
    \AttributeTok{fill =} \StringTok{"Género"}
\NormalTok{  ) }\SpecialCharTok{+}
  \FunctionTok{theme\_minimal}\NormalTok{()}
\end{Highlighting}
\end{Shaded}

\pandocbounded{\includegraphics[keepaspectratio]{Reporte-Analisis-Students-Performance_files/figure-latex/unnamed-chunk-3-3.pdf}}

Conclusiones: De acuerdo con los hallazgos del análisis, se
identificaron tres factores clave que influyen en el desempeño
estudiantil: el tipo de almuerzo, el nivel educativo de los padres y las
diferencias por género en áreas específicas. Para optimizar el
rendimiento académico en la institución, se recomienda implementar las
siguientes estrategias:

En primer lugar, dado que los estudiantes con almuerzo ``standard''
obtienen mejores resultados, se sugiere evaluar y fortalecer los
programas de alimentación escolar, especialmente para aquellos en
situación vulnerable. Esto podría incluir la ampliación de becas de
comedor, la mejora nutricional de los menús o la implementación de
desayunos escolares, asegurando que todos los alumnos tengan acceso a
una alimentación equilibrada. Estudios respaldan que una nutrición
adecuada está directamente vinculada a la concentración y el rendimiento
cognitivo.

En segundo lugar, la correlación entre el nivel educativo de los padres
y el éxito académico de los estudiantes indica la necesidad de programas
que involucren a las familias en el proceso educativo. Se podrían
organizar talleres para padres sobre técnicas de apoyo en el hogar,
tutorías conjuntas o acceso a recursos educativos digitales. Además,
para abordar las diferencias por género ---como el menor desempeño de
las mujeres en matemáticas---, se recomienda fomentar mentorías con
enfoque de género, clubes de ciencias dirigidos a mujeres y capacitación
docente en metodologías inclusivas que reduzcan estereotipos.

Finalmente, un enfoque integral que combine políticas nutricionales,
participación familiar y equidad de género permitirá no solo mejorar los
resultados académicos, sino también reducir brechas socioeconómicas y
promover un ambiente educativo más inclusivo y equitativo. La
institución podría medir el impacto de estas intervenciones mediante
evaluaciones periódicas, ajustando las estrategias según los resultados
obtenidos.

\end{document}
